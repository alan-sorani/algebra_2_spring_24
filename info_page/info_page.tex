\documentclass{article}
\usepackage{graphicx} % Required for inserting images

%-------------------------------
%           HYPERLINKS
%-------------------------------

\usepackage{hyperref}

%-------------------------------
%            HEBREW
%-------------------------------

\usepackage{hebrewdoc}

%-------------------------------
%            LISTS
%-------------------------------

\usepackage{enumitem}

%-------------------------------
%            TITLE
%-------------------------------

\title{דף מידע \\ אלגברה ב' - 104168 \\ אביב 2024}
\date{}

\begin{document}

\maketitle

\section*{סגל הקורס}

\begin{description}
\item[מרצה:]
מיכה שגיב
-
אמדו 922
-
\textenglish{\href{mailto:sageevm@technion.ac.il}{sageevm@technion.ac.il}}
\\
\emph{שעת קבלה:}
יום א' 14:30-15:20
\item[מתרגל אחראי:] 
אלן סורני
-
\textenglish{\href{mailto:elad.tzorani@campus.technion.ac.il}{elad.tzorani@campus.technion.ac.il}}
\\
\emph{שעת קבלה:}
יום ה' 12:30-13:20
באולמן 505
\item[מתרגל:] 
ראובן יעקב
-
אמדו 722
-
\textenglish{\href{mailto:rubenj@technion.ac.il}{rubenj@technion.ac.il}}
\\
\emph{שעת קבלה:}
יום ג' 14:30-15:20
\end{description}

\section*{מבנה הציון}

הציון הסופי בקורס יקבע באופן הבא.

\begin{itemize}
\item[-] \emph{15\%}
מגן עבור שיעורי בית. רכיב זה יקבע לפי ממוצע 4 גיליונות שיעורי הבית הטובים ביותר מתוך 6.
\\
המגן ילקח בחשבון בתנאי שהוא גבוה מציון המבחן, ובתנאי שציון זה הוא לפחות 50.
\item[-] \emph{85-100\%}
בחינה סופית.
\end{itemize}

\section*{מידע כללי}

\begin{itemize}
\item[-]
יפורסם במהלך הקורס בוחן דמה לתרגול עצמי, ללא בדיקה או ציון.
\item[-]
פניות בנושאי ניהול הקורס ושיעורי בית יש להפנות לאלן המתרגל האחראי.
\end{itemize}

\pagebreak

\section*{סילבוס}
\begin{enumerate}
\item הדטרמיננטה: הגדרה ותכונות. מטריצה מצורפת וכלל קרמר.
\item סכומים ישרים, הטלות, תת־מרחבים שמורים וצמצום של אופרטור לינארי.
\item אופרטורים נילפוטנטיים ואינדקס הנילפוטנטיות. צורת ז'ורדן.
\item משפט קיילי המילטון ומשפט הפירוק הפרימרי.
\item מרחבי מכפלה פנימית, המשלים הניצב ותהליך גרם־שמידט. הטלות אורתוגונליות.
\item האופרטור הצמוד, אופרטורים נורמליים, אופרטורים צמודים לעצמם, אופרטורים אוניטריים ואיזומטריות.
\item משפט הפירוק הספקטרלי ולכסון אורתוגונלי. אופרטורים חיוביים ומשפט הפירוק הפולרי. \textenglish{SVD}.
\item תבניות בילינאריות ותבניות ריבועיות. חפיפת מטריצות. משפט סילבסטר.
\item אלגברה מולטילינארית ומכפלות טנזוריות.
\end{enumerate}

\section*{ספרות}
\begin{itemize}
\item[-] סיכומי הרצאות של ניר לזרוביץ' (נמצאים במודל)
\item[-] \textenglish{Algebra (2nd Edition) - M.~Artin}
\item[-] \textenglish{Linear Algebra - K.~Hoffman and R.F.~Kunze}
\item[-] \textenglish{Linear Algebra Done Right - S.~Axler}
\end{itemize}

\end{document}