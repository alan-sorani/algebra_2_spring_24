\documentclass[a4paper,10pt,twoside,openany]{article}

\usepackage[lang=hebrew]{maths}
\usepackage{hebrewdoc}
\usepackage{stylish}
\usepackage{lipsum}
\let\bs\blacksquare

\setlength{\parindent}{0pt}

%%%%%%%%%%%%
% Styling %
%%%%%%%%%%%%

\usepackage{enumitem}

%%%%%%%%%%%%%
% Counters  %
%%%%%%%%%%%%%

\setcounter{section}{0}     
            
%BIBLIOGRAPHY
\usepackage[
backend=biber,
style=alphabetic,
]{biblatex}
\addbibresource{bibliography.bib} %Imports bibliography file

%%%%%%%%%%
% Title  %
%%%%%%%%%%
\title{
אלגברה ב' - תרגילי הכנה לקורס \\
חזרה על מטריצות מייצגות, הדטרמיננטה, ערכים עצמיים ולכסון
\\
\vspace{1cm}
\large{לא להגשה}
}
\date{}

\begin{document}
\maketitle

\section*{סימונים}

\begin{itemize}
\item[-]
נסמן ב־%
$\brs{n} \coloneqq \set{1, 2, 3, \ldots, n}$
את אוסף המספרים הטבעיים החיוביים עד
$n$.

\item[-]
משמעות הסימון
$\sum_{i \in \brs{n}} a_i$
היא סכום האיברים
$a_1, \ldots, a_n$.

\item[-]
משמעות הסימון
$\prod_{i \in \brs{n}} a_i$
היא מכפלת האיברים
$a_1, \ldots, a_n$.

\item[-]
עבור פולינום
\[\text{,} p\prs{x} = \sum_{i = 0}^n a_i x^i \in \mbb{F}\brs{x}\]
הנגזרת הפורמלית
$p'\prs{x}$
היא
\[\text{.} p'\prs{x} = \sum_{i=0}^{n-1} a_{i+1} \prs{i+1} x^i \in \mbb{F}\brs{x}\]
אם
$\mbb{F} = \mbb{R}$
זאת אותה נגזרת מהקורס אינפי 1.

\item[-]
עבור כל שני ביטויים מתמטיים
$x,y$
נסמן
\[\text{.}\delta_{x,y} = \begin{cases} 1 & x = y \\ 0 & x \neq y \end{cases}\]
סימון זה נקרא
\emph{הדלתא של קרונקר}.

\item[-]
נסמן ב־%
$\Mat_{m \times n}\prs{\mbb{F}}$
את מרחב המטריצות עם
$m$
שורות ו־%
$n$
עמודות עם מקדמים בשדה
$\mbb{F}$.
נסמן לעתים
$\Mat_{n}\prs{\mbb{F}} = \Mat_{n \times n}\prs{\mbb{F}}$.

\item[-]
עבור מרחבים וקטוריים
$V,W$
מעל אותו שדה
$\mbb{F}$
נסמן ב־%
$\hom_{\mbb{F}}\prs{V,W}$
את אוסף ההעתקות הלינאריות מ־%
$V$
ל־%
$W$.
נסמן גם
$\End_{\mbb{F}}\prs{V} = \hom_{\mbb{F}}\prs{V,V}$.
\end{itemize}

\section{מטריצות מייצגות}

\begin{definition}[מטריצה מייצגת]
יהיו
$V,W$
מרחבים וקטורים סוף־מימדיים מעל אותו שדה
$\mbb{F}$
עם בסיסים
$B,C$
בהתאמה, ונסמן
\[\text{.} B = \prs{v_1, \ldots, v_n}\]
נסמן גם
$n \coloneqq \dim\prs{V}$
ו־%
$m \coloneqq \dim\prs{W}$.
עבור
$T \in \Hom_{\mbb{F}}\prs{V,W}$
נגדיר
\[\text{.} \brs{T}^B_C = \pmat{\vert & & \vert \\ \brs{T\prs{v_1}}_C & \cdots & \brs{T\prs{v_n}}_C \\ \vert & & \vert} \in \Mat_{m \times n}\prs{\mbb{F}}\]
\end{definition}

\begin{exercise}
חשבו את המטריצות המייצגות של ההעתקות הבאות לפי הבסיסים הנתונים.

\begin{enumerate}
\item ההעתקה
\begin{align*}
T \colon \mbb{R}^3 &\to \mbb{R}^3 \\
\pmat{x \\ y \\ z} &\mapsto \pmat{x+y \\ y+z \\ x+z}
\end{align*}
לפי הבסיס
\begin{align*}
\text{.} B = \prs{\pmat{1 \\ 0 \\ 0}, \pmat{0 \\ 1 \\ 1}, \pmat{0 \\ 1 \\ -1}}
\end{align*}
\item ההעתקה
\begin{align*}
T \colon \mbb{C}_3\brs{x} &\to \mbb{C}_3\brs{x} \\
p\prs{x} &\mapsto p\prs{x+1}
\end{align*}
לפי הבסיס הסטנדרטי
\begin{align*}
\prs{1, x, x^2, x^3}
\end{align*}
של מרחב הפולינומים ממעלה לכל היותר
$3$
עם מקדמים מרוכבים.
\item ההעתקה
\begin{align*}
T \colon \Mat_{2\times 2}\prs{\mbb{R}} &\to \Mat_{2 \times 2}\prs{\mbb{R}}\\
A &\mapsto A^t
\end{align*}
לפי הבסיס
\begin{align*}
\text{.} B = \prs{\pmat{1 & 0 \\ 0 & 0}, \pmat{1 & 1 \\ 0 & 0}, \pmat{1 & 1 \\ 1 & 0}, \pmat{1 & 1 \\ 1 & 1}}
\end{align*}
\item ההעתקה
\begin{align*}
T_A \colon \Mat_{n \times n}\prs{\mbb{C}} &\to \Mat_{n \times n}\prs{\mbb{C}} \\
B &\mapsto AB
\end{align*}
עבור מטריצה
$A \in \Mat_{n \times n}\prs{\mbb{C}}$
ולפי הבסיס הסטנדרטי
\begin{align*}
E = \prs{E_{1,1}, \ldots, E_{1,n}, E_{2,1}, \ldots, E_{2,n}, \ldots, E_{n,1}, \ldots, E_{n,n}}
\end{align*}
כאשר
$E_{i,j}$
מטריצה שמקדמיה אפסים חוץ מ־$1$ במיקום ה־$\prs{i,j}$. כלומר,
\begin{align*}
\text{.} \prs{E_{i,j}}_{k,\ell} = \delta_{\prs{i,j}, \prs{k,\ell}} \coloneqq
\begin{cases}
0 & \prs{i,j} = \prs{k,\ell} \\
1 & \prs{i,j} \neq \prs{k,\ell}
\end{cases}
\end{align*}
\item ההעתקה
\begin{align*}
T \colon V &\to V \\
b_i &\mapsto
\begin{cases}
b_{i+1} & i < n \\
b_1 & i = n
\end{cases}
\end{align*}
עבור בסיס
\begin{align*}
\text{,} B = \prs{b_1, \ldots, b_n}
\end{align*}
לפי הבסיס
$B$.
\end{enumerate}
\end{exercise}

\begin{exercise}
בכל אחד מהסעיפים הבאים, מיצאו העתקה לינארית
$T \colon V \to V$
עבורה
$\brs{T}_B = A$.
\begin{enumerate}
\item
\begin{align*}
V &= \mbb{R}^3 \\
B &= \prs{e_2, e_3, e_1} \\
A &= \pmat{1 & 2 & 3 \\ 4 & 5 & 6 \\ 7 & 8 & 9}
\end{align*}
\item
\begin{align*}
V &= \mbb{R}_2\brs{x} \\
B &= \prs{1, x, x^2} \\
A &= J_3\prs{1} \coloneqq \pmat{1 & 1 & 0 \\ 0 & 1 & 1 \\ 0 & 0 & 1}
\end{align*}
\item
\begin{align*}
V &= \Mat_{2 \times 2}\prs{\mbb{C}} \\
B &= \prs{E_{1,1}, E_{1,2}, E_{2,1}, E_{2,2}} \\
A &= \pmat{0 & 1 & 0 & 0 \\ 1 & 0 & 0 & 0 \\ 0 & 0 & 0 & 1 \\ 0 & 0 & 1 & 0}
\end{align*}

\item
$V$
מרחב הפונקציות
$\brs{4} \to \mbb{R}$.
הבסיס
$B$
הוא
$\prs{\chi_1, \chi_2, \chi_3, \chi_4}$
כאשר
$\chi_i\prs{j} = \delta_{i,j}$.
המטריצה
$A$
היא
\[\text{.} A = \pmat{1 & 1 & 0 & 0 \\ 0 & 1 & 0 & 0 \\ 0 & 0 & 1 & 1 \\ 0 & 0 & 0 & 1}\]
\end{enumerate}
\end{exercise}

\begin{exercise}
הוכיחו/הפריכו את הטענה הבאה:
תהי
$T \colon V \to V$
העתקה לינארית ויהיו שני בסיסים
$B,C$
של
$V$
עבורם
$\brs{T}_B = \brs{T}_C$.
אז
$B = C$.
\end{exercise}

\section{הדטרמיננטה}

\begin{definition}[דטרמיננטה]
תהי
$A \in \Mat_{n \times n}\prs{\mbb{F}}$
מטריצה עם מקדמים בשדה
$\mbb{F}$.
נגדיר את
\emph{הדטרמיננטה}
$\det\prs{A}$
של
$A$
באופן הרקורסיבי הבא.

תהי
$M_{i,j}$
הדטרמיננטה של המטריצה המתקבלת מ־$A$ לאחר הוצאת השורה ה־$i$ והעמודה ה־$j$ של $A$.
נקרא למספר זה
\emph{המינור}
ה־$\prs{i,j}$ של
$A$.
הדטרמיננטה של
$A$
מוגדרת על ידי
\begin{align*}
\text{.} \det\prs{A} &\coloneqq \sum_{i,j \in \brs{n}} \prs{-1}^{i+j} a_{i,j} M_{i,j}
\end{align*}
\end{definition}

\begin{theorem}
תהיינה
$A, B, C \in \Mat_{n \times n}\prs{\mbb{F}}$
ונניח כי
$C$
הפיכה.
מתקיימות התכונות הבאות.
\begin{enumerate}
\item $\det\prs{AB} = \det\prs{A} \cdot \det\prs{B}$
\item $\det\prs{C^{-1}} = \prs{\det\prs{C}}^{-1}$
\end{enumerate}
\end{theorem}

\begin{exercise}\label{exercise:matrix-determinant}
חשבו את הדטרמיננטות של המטריצות הבאות.

\begin{enumerate}
\item \[A = \pmat{1 & 2 \\ 3 & 4} \in \Mat_{2 \times 2}\prs{\mbb{R}}\]
\item \[B = \pmat{\cos\prs{\theta} & -\sin\prs{\theta} \\ \sin\prs{\theta} & \cos\prs{\theta}} \in \Mat_{2 \times 2}\prs{\mbb{C}}\]
עבור
$\theta \in \mbb{R}$.
\item \[C = \pmat{a & b \\ c & d} \in \Mat_{n \times n}\prs{\mbb{F}}\]
עבור סקלרים
$a,b,c,d \in \mbb{F}$.
\item \[D = \pmat{0 & 1 & 0 \\ 0 & 0 & 1 \\ 1 & 0 & 0} \in \Mat_{3 \times 3}\prs{\mbb{C}}\]
\item \[E = \pmat{0 & 1 & 0 \\ 1 & 0 & 0 \\ 0 & 0 & 1} \in \Mat_{3 \times 3}\prs{\mbb{C}}\]
\item \[F = \pmat{A & 0_{2 \times 2} \\ 0_{2 \times 2} & B} \in \Mat_{4 \times 4}\prs{\mbb{F}}\]
עבור
$A,B \in \Mat_{2 \times 2}\prs{\mbb{F}}$.
\end{enumerate}
\end{exercise}

\begin{proposition}[מעבר בסיס]\label{proposition:change-of-basis}
תהי
$T \colon V \to V$
העתקה לינארית ויהיו
$B,C$
שני בסיסים של
$V$.
נסמן
$M^B_C = \brs{\id_V}^B_C$
ונקרא לה
\emph{מטריצת מעבר}
בין הבסיסים
$B,C$.
מתקיים
\begin{enumerate}
\item \[M^C_B = \prs{M^B_C}^{-1}\]
\item \[\brs{T}_B = \prs{M^B_C}^{-1} \brs{T}_C M^B_C\]
\end{enumerate}
\end{proposition}

\begin{remark}
המטריצה
$M^B_C$
מקיימת
$M^B_C \brs{v}_B = \brs{v}_C$
לכל וקטור
$v \in V$.
בכל זאת, יש שקוראים לה
"מטריצת מעבר מ־%
$C$
ל־%
$B$".
לכן לא נקרא לה
"מטריצת מעבר מבסיס אחד לאחר"
אלא רק מטריצת מעבר בין שני הבסיסים.
\end{remark}

\begin{exercise}
תהי
$T \colon V \to V$
העתקה לינארית, ויהי
$B$
בסיס של
$V$.
נגדיר את הדטרמיננטה של
$T$
על ידי
\[\text{.}\det\prs{T} \coloneqq \det\prs{\brs{T}_B}\]
הראו שהדטרמיננטה של
$T$
מוגדרת היטב. כלומר, הראו שאם
$B'$
בסיס נוסף של
$V$
מתקיים
\[\text{.} \det\prs{\brs{T}_B} = \det\prs{\brs{T}_{B'}}\]
\end{exercise}

\section{ערכים ווקטורים עצמיים}

\begin{definition}[ערכים ווקטורים עצמיים]
תהי
$T \colon V \to V$
העתקה לינארית מעל שדה
$\mbb{F}$.
סקלר
$\lambda \in \mbb{F}$
נקרא
\emph{ערך עצמי של
$T$}
אם קיים
$v \in V$
עבורו
$T\prs{v} = \lambda v$.
וקטור
$v \in V$
כזה נקרא
\emph{וקטור עצמי של
$T$}
עם ערך עצמי
$\lambda$.

תהי
$A \in \Mat_{n \times n}\prs{\mbb{F}}$
ונסמן
\begin{align*}
T_A \colon \mbb{F}^n &\to \mbb{F}^n \\
\text{.} \hphantom{lalala} v &\mapsto Av
\end{align*}
ערך/וקטור
עצמי של
$A$
הוא
ערך/וקטור
עצמי של
$T_A$.
\end{definition}

\begin{exercise}
מצאו את הערכים העצמיים של ההעתקות הבאות.

\begin{enumerate}
\item
ההעתקה
\begin{align*}
T \colon \mbb{R}^3 &\to \mbb{R}^3 \\
\pmat{x \\ y \\ z} &\mapsto \pmat{x \\ x+y \\ x + y + z}
\end{align*}
\item
ההעתקה
\begin{align*}
T \colon \mbb{R}^2 &\to \mbb{R}^2 \\
\pmat{x \\ y} &\mapsto \pmat{-y \\ x}
\end{align*}
\item
ההעתקה
\begin{align*}
T \colon \mbb{C}^2 &\to \mbb{C}^2 \\
\pmat{x \\ y} &\mapsto \pmat{-y \\ x}
\end{align*}
\item ההעתקה
\begin{align*}
T \colon \mbb{C}_3\brs{x} &\to \mbb{C}_3\brs{x} \\
p\prs{x} &\mapsto p\prs{x+1}
\end{align*}
\item ההעתקה
\begin{align*}
T \colon \mbb{F}_n\brs{x} &\to \mbb{F}_n\brs{x} \\
p\prs{x} &\mapsto p'\prs{x}
\end{align*}
כאשר
$p'\prs{x}$
הנגזרת הפורמלית של
$p\prs{x}$.
\item ההעתקה
\begin{align*}
T \colon \Mat_{2\times 2}\prs{\mbb{R}} &\to \Mat_{2 \times 2}\prs{\mbb{R}}\\
A &\mapsto A^t
\end{align*}
\item ההעתקה
\begin{align*}
T \colon V &\to V \\
b_i &\mapsto
\begin{cases}
b_{i+1} & i < n \\
b_1 & i = n
\end{cases}
\end{align*}
עבור בסיס
\begin{align*}
\text{,} B = \prs{b_1, \ldots, b_n}
\end{align*}
כאשר
$V$
מרחב וקטורי מעל שדה
$\mbb{F}$.
\end{enumerate}
\end{exercise}

\begin{theorem}
תהי
$A \in \Mat_{n \times n}\prs{\mbb{F}}$.
\begin{enumerate}
\item
הערכים העצמיים של $A$
הם שורשי הפולינום המתוקן
$p_A\prs{x} \coloneqq \det\prs{xI_{n \times n} - A}$.
\item
אם ל־%
$p_A$
יש
$n$
שורשים
$\lambda_1, \ldots, \lambda_n$
כולל ריבויים, העכבה של $A$ היא סכום השורשים והדטרמיננטה של
$A$
היא מכפלת השורשים. כלומר,
\begin{align*}
\tr\prs{A} &= \sum_{i \in \brs{n}} \lambda_i \\
\text{.} \det\prs{A} &= \prod_{i \in \brs{n}} \lambda_i
\end{align*}
\end{enumerate}
\end{theorem}

\begin{exercise}
מצאו את הערכים העצמיים של המטריצות הבאות. עבור כל אחת מהמטריצות
$A$
עד
$F$
שיש לה מספר ערכים עצמיים ששווה לגודל המטריצה,
וודאו שמכפלת הערכים העצמיים שווה לדטרמיננטה שחישבתם בתרגיל
\ref{exercise:matrix-determinant}.
\begin{enumerate}
\item \[A = \pmat{1 & 2 \\ 3 & 4} \in \Mat_{2 \times 2}\prs{\mbb{R}}\]
\item \[B = \pmat{\cos\prs{\theta} & -\sin\prs{\theta} \\ \sin\prs{\theta} & \cos\prs{\theta}} \in \Mat_{2 \times 2}\prs{\mbb{C}}\]
עבור
$\theta \in \mbb{R}$.
\item \[B' = \pmat{\cos\prs{\theta} & -\sin\prs{\theta} \\ \sin\prs{\theta} & \cos\prs{\theta}} \in \Mat_{2 \times 2}\prs{\mbb{R}}\]
עבור
$\theta \in \mbb{R}$.
\item \[C = \pmat{a & b \\ c & d} \in \Mat_{n \times n}\prs{\mbb{F}}\]
עבור סקלרים
$a,b,c,d \in \mbb{F}$.
\item \[D = \pmat{0 & 1 & 0 \\ 0 & 0 & 1 \\ 1 & 0 & 0} \in \Mat_{3 \times 3}\prs{\mbb{C}}\]
\item \[E = \pmat{0 & 1 & 0 \\ 1 & 0 & 0 \\ 0 & 0 & 1} \in \Mat_{3 \times 3}\prs{\mbb{C}}\]
\item \[F = \pmat{A & 0_{2 \times 2} \\ 0_{2 \times 2} & B} \in \Mat_{4 \times 4}\prs{\mbb{F}}\]
עבור
$A,B \in \Mat_{2 \times 2}\prs{\mbb{F}}$.
\end{enumerate}
\end{exercise}

\section{לכסון}

\begin{definition}[לכסינות]
העתקה לינארית
$T \colon V \to V$
נקראת
\emph{לכסינה}
אם קיים בסיס
$B$
של
$V$
עבורו
$\brs{T}_B$
מטריצה אלכסונית.
בסיס
$B$
כזה נקרא
\emph{בסיס מלכסן}.

מטריצה
$A \in \Mat_{n \times n}\prs{\mbb{F}}$
נקראת
\emph{לכסינה}
אם
$T_A$
לכסינה.
\end{definition}

\begin{exercise}
הראו כי
$T \colon V \to V$
לכסינה אם ורק אם קיים בסיס
$B$
של
$V$
שאיבריו עם וקטורים עצמיים של
$T$.
הראו כי במקרה זה הערכים העצמיים של
$T$
הם הערכים על האלכסון של
$\brs{T}_B$.
\end{exercise}

\begin{exercise}
הראו כי
$A \in \Mat_{n \times n}\prs{\mbb{F}}$
לכסינה אם ורק אם קיימת מטריצה הפיכה
$P \in \Mat_{n \times n}\prs{\mbb{F}}$
עבורה
$P^{-1} A P$
מטריצה אלכסונית.

\emph{רמז:}
העזרו בטענה
\ref{proposition:change-of-basis}.
\end{exercise}

\begin{exercise}
עבור כל אחת מההעתקות הבאות קבעו האם ההעתקה לכסינה ואם כן מצאו בסיס מלכסן. הוכיחו את טענותיכן.
\begin{enumerate}
\item
ההעתקה
\begin{align*}
T \colon \mbb{R}^3 &\to \mbb{R}^3 \\
\pmat{x \\ y \\ z} &\mapsto \pmat{x \\ x+y \\ x + y + z}
\end{align*}
\item
ההעתקה
\begin{align*}
T \colon \mbb{R}^2 &\to \mbb{R}^2 \\
\pmat{x \\ y} &\mapsto \pmat{-y \\ x}
\end{align*}
\item
ההעתקה
\begin{align*}
T \colon \mbb{C}^2 &\to \mbb{C}^2 \\
\pmat{x \\ y} &\mapsto \pmat{-y \\ x}
\end{align*}
\item ההעתקה
\begin{align*}
T \colon \mbb{C}_3\brs{x} &\to \mbb{C}_3\brs{x} \\
p\prs{x} &\mapsto p\prs{x+1}
\end{align*}
\item ההעתקה
\begin{align*}
T \colon \mbb{F}_n\brs{x} &\to \mbb{F}_n\brs{x} \\
p\prs{x} &\mapsto p'\prs{x}
\end{align*}
כאשר
$p'\prs{x}$
הנגזרת הפורמלית של
$p\prs{x}$.
\item ההעתקה
\begin{align*}
T \colon \Mat_{2\times 2}\prs{\mbb{R}} &\to \Mat_{2 \times 2}\prs{\mbb{R}}\\
A &\mapsto A^t
\end{align*}
\item ההעתקה
\begin{align*}
T \colon V &\to V \\
b_i &\mapsto
\begin{cases}
b_{i+1} & i < n \\
b_1 & i = n
\end{cases}
\end{align*}
עבור בסיס
\begin{align*}
\text{,} B = \prs{b_1, \ldots, b_n}
\end{align*}
כאשר
$V$
מרחב וקטורי מעל שדה
$\mbb{F}$.
\end{enumerate}
\end{exercise}

\begin{exercise}
עבור כל אחת מהמטריצות הבאות $X$ קבעו האם המטריצה לכסינה ואם כן מצאו מטריצה
$P$
עבורה
$P^{-1} X P$
מטריצה אלכסונית.
הוכיחו את טענותיכן.

\begin{enumerate}
\item \[A = \pmat{1 & 2 \\ 3 & 4} \in \Mat_{2 \times 2}\prs{\mbb{R}}\]
\item \[B = \pmat{\cos\prs{\theta} & -\sin\prs{\theta} \\ \sin\prs{\theta} & \cos\prs{\theta}} \in \Mat_{2 \times 2}\prs{\mbb{C}}\]
עבור
$\theta \in \mbb{R}$.
\item \[B' = \pmat{\cos\prs{\theta} & -\sin\prs{\theta} \\ \sin\prs{\theta} & \cos\prs{\theta}} \in \Mat_{2 \times 2}\prs{\mbb{R}}\]
עבור
$\theta \in \mbb{R}$.
\item \[D = \pmat{0 & 1 & 0 \\ 0 & 0 & 1 \\ 1 & 0 & 0} \in \Mat_{3 \times 3}\prs{\mbb{C}}\]
\item \[E = \pmat{0 & 1 & 0 \\ 1 & 0 & 0 \\ 0 & 0 & 1} \in \Mat_{3 \times 3}\prs{\mbb{C}}\]
\item \[F = \pmat{A & 0_{2 \times 2} \\ 0_{2 \times 2} & B} \in \Mat_{4 \times 4}\prs{\mbb{F}}\]
עבור
$A,B \in \Mat_{2 \times 2}\prs{\mbb{F}}$.
הפרידו למקרים כתלות במטריצות
$A,B$.
\item \[J\prs{\lambda} = \pmat{\lambda & 1 \\ 0 & \lambda} \in \Mat_{2 \times 2}\prs{\mbb{C}}\]
עבור
$\lambda \in \mbb{C}$.
\item \[K = \pmat{1 & 1 \\ 0 & -1} \in \Mat_{2 \times 2}\prs{\mbb{C}}\]
\end{enumerate}
\end{exercise}

\begin{exercise}
תהיינה
$A,B \in \Mat_{n \times n}\prs{\mbb{F}}$
לכסינות. הפריכו את הטענות הבאות.
\begin{enumerate}
\item $A+B$ לכסינה.
\item $AB$ לכסינה.
\end{enumerate}
\emph{רמז:}
היעזרו במטריצות
$J\prs{0}, J\prs{1}, K$
מהסעיף הקודם.
\end{exercise}

\begin{exercise}
הוכיחו שאם למטריצה יש ערך עצמי יחיד, היא לכסינה אם ורק אם היא סקלרית, כלומר אם ורק אם היא מהצורה
$\lambda I_{n \times n}$.
\end{exercise}

\end{document}