\documentclass[a4paper,10pt,twoside,openany]{article}

\usepackage[lang=hebrew]{maths}
\usepackage{hebrewdoc}
\usepackage{stylish}
\usepackage{lipsum}
\let\bs\blacksquare

\setlength{\parindent}{0pt}

%%%%%%%%%%%%
% Styling %
%%%%%%%%%%%%

\usepackage{enumitem}

%%%%%%%%%%%%%
% Counters  %
%%%%%%%%%%%%%

\setcounter{section}{1}     
            
%BIBLIOGRAPHY
\usepackage[
backend=biber,
style=alphabetic,
]{biblatex}
\addbibresource{bibliography.bib} %Imports bibliography file

%%%%%%%%%%
% Title  %
%%%%%%%%%%
\title{
אלגברה ב' - גיליון תרגילי בית 4 \\
גרם־שמידט ואופרטורים צמודים
\\
\vspace{1cm}
\large{תאריך הגשה: 9.8.2024}
}
\date{}

\begin{document}
\maketitle

\begin{exercise}%1
יהי
$V = \mbb{R}_2\brs{x}$
ותהיינה
\begin{align*}
\trs{f,g}_1 &= \int_0^1 f\prs{x} g\prs{x} \diff x \\
\trs{f,g}_2 &= f\prs{-1} g\prs{-1} + f\prs{0} g\prs{0} + f\prs{1} g\prs{1}
\end{align*}
שתי מכפלות פנימיות על
$V$.
יהי
\[\text{.} W = \set{f \in V}{f\prs{x} = f\prs{-x}} \leq V\]
\begin{enumerate}
\item 
מיצאו בסיס
$B = \prs{w_1, \ldots, w_m}$
של
$W$
והשלימו אותו לבסיס
$C$
של
$V$.
בצעו את תהליך גרם־שמידט על
$C$
לפי כל אחת מהמכפלות הפנימיות כדי לקבל בסיסים אורתונורמליים לפיהן.

\item
היעזרו בבסיסים שמצאתן בסעיף הקודם כדי למצוא
את
$W^\perp$
לפי כל אחת מהמכפלות הפנימיות.

\item
מיצאו את ההטלה האורתוגונלית
$P_W$
על
$W$
לפי כל אחת מהמכפלות הפנימיות.

\item
יהי
$f\prs{x} = 1 + x$.
מיצאו את המרחק של
$f$
מ־%
$W$
לפי כל אחת מהמכפלות הפנימיות.
\end{enumerate}
\end{exercise}

\begin{exercise}%2
יהי
$V$
מרחב מכפלה פנימית סוף־מימדי מעל
$\mbb{C}$,
ויהי
$T \in \End_{\mbb{C}}\prs{V}$.
הראו כי קיים בסיס אורתונורמלי
$B$
עבורו
$\brs{T}_B$
משולשת עליונה.

\textbf{רמז:}
היעזרו במשפטי ז'ורדן וגרם־שמידט.
\end{exercise}

\begin{exercise}%3
יהי
$V = M_2\prs{\mbb{R}}$
עם הבסיס
\[\text{.} B = \prs{\pmat{1 & 1 \\ 0 & 0}, \pmat{1 & 2 \\ 0 & 0}, \pmat{1 & 0 \\ 1 & 0}, \pmat{0 & 1 \\ 0 & 1}}\]
מיצאו מכפלה פנימית על
$V$
לפיה
$B$
בסיס אורתונורמלי.

\textbf{רמז:}
כל המכפלות פנימיות על
$V$
הן מהצורה
\[\text{,} \trs{u,v}_C \coloneqq \trs{\brs{u}_C, \brs{v}_C}_{\mrm{std}}\]
עבור
$C$
בסיס של
$V$.
\end{exercise}

\begin{exercise}
יהי
$V = \Mat_n\prs{\mbb{R}}$
עם המכפלה הפנימית
$\trs{X,Y} = \tr\prs{Y^t X}$,
ותהי
$B \in V$.
נגדיר אופרטור
\begin{align*}
\Phi_B \colon V &\to V \\
\text{.} \hphantom{lalala} A &\mapsto BA
\end{align*}
\begin{enumerate}
\item
חשבו את
$\Phi_B^*$.
\item
עבור אילו מטריצות
$B$
האופרטור
$\Phi_B$
נורמלי?
\item
עבור אילו מטריצות
$B$
האופרטור
$\Phi_B$
צמוד לעצמו?
\item
עבור אילו מטריצות
$B$
האופרטור
$\Phi_B$
אורתוגונלי?
\end{enumerate}
\end{exercise}

\begin{exercise}
יהי
$V = \Mat_2\prs{\mbb{R}}$
עם המכפלה הפנימית
$\trs{X,Y} = \tr\prs{Y^t X}$,
ויהי
$T \in \End_{\mbb{R}}\prs{V}$
המוגדר על ידי
\[\text{.} T\pmat{a & b \\ c & d} = \pmat{16 a & 4b -6c \\ -6b + 13c & 16d}\]
מיצאו בסיס אורתונורמלי
$B$
של
$V$
עבורו
$\brs{T}_B$
אלכסונית, ומיצאו אופרטור
$S \in \End_{\mbb{R}}\prs{V}$
צמוד לעצמו עבורו
$S^2 = T$.
\end{exercise}

\begin{exercise}
יהי
$V$
מרחב מכפלה פנימית מרוכב ממימד סופי, יהי
$T \in \End_{\mbb{C}}\prs{V}$
נורמלי, ונניח כי
$3,4$
ערכים עצמיים של
$T$.
הראו שיש
$v \in V$
עבורו
$\norm{v} = \sqrt{2}$
וגם
$\norm{Tv} = 5$.
\end{exercise}

\begin{exercise}
יהי
$V$
מרחב מכפלה פנימית מרוכב ממימד סופי, יהי
$T \in \End_{\mbb{C}}\prs{V}$
ויהי
$a \in \mbb{C}$
עבורו
$\abs{a} \neq 1$.

\begin{enumerate}
\item הראו כי אם
$T^* = aT$
אז
$T = 0$.

\item נניח כי
$T$
נורמלי, ויהי
$S = T - a T^*$.
הוכיחו כי
$\ker\prs{T} = \ker\prs{S}$.
\end{enumerate}
\end{exercise}

\end{document}