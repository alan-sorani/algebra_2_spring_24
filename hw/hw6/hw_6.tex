\documentclass[a4paper,10pt,twoside,openany]{article}

\usepackage[lang=hebrew]{maths}
\usepackage{hebrewdoc}
\usepackage{stylish}
\usepackage{lipsum}
\let\bs\blacksquare

\setlength{\parindent}{0pt}

%%%%%%%%%%%%
% Styling %
%%%%%%%%%%%%

\usepackage{enumitem}

%%%%%%%%%%%%%
% Counters  %
%%%%%%%%%%%%%

\setcounter{section}{1}     
            
%BIBLIOGRAPHY
\usepackage[
backend=biber,
style=alphabetic,
]{biblatex}
\addbibresource{bibliography.bib} %Imports bibliography file

%%%%%%%%%%
% Title  %
%%%%%%%%%%
\title{
אלגברה ב' - תרגילי חזרה למבחן
\\
\vspace{1cm}
\large{תאריך הגשה: 7.9.2024}
}
\date{}

\begin{document}
\maketitle

\begin{exercise}
נגיד כי וקטור
$\pmat{a \\ b \\ c} \in \mbb{R}_3\brs{x}$
עם מקדמים שלמים הינו
\emph{וקטור פיתגורי}
אם
$\prs{a,b,c}$
שלשה פיתגורית, כלומר
$a,b,c$
שלמים חיוביים ומתקיים
\[\text{.} a^2 + b^2 = c^2\]

תהי
\[A = \pmat{1 & -2 & 2 \\ 2 & -1 & 2 \\ 2 & -2 & 3} \in \Mat_3\prs{\mbb{R}}\]
ויהי
\[\text{.} v_0 = \pmat{3 \\ 4 \\ 5} \in \mathbb{R}^3\]

\begin{enumerate}
\item
הראו כי
$v_0$
וקטור פיתגורי וכי אם
$v \in \mbb{R}^3$
וקטור פיתגורי אז
$Av$
וקטור פיתגורי.

\item
הסיקו כי
$A^k v_0$
וקטור פיתגורי לכל
$k$.

\item
חשבו את
$A^k v_0$
לכל
$k \in \mbb{N}$,
ומצאו וקטור פיתגורי
$\pmat{a \\ b \\ c}$
עבורו
$a = 2025$.
\end{enumerate}
\end{exercise}

\begin{exercise}
תהי
\[\text{.} A = \pmat{0 & 1 & 0 & 1 \\ 1 & 0 & 1 & 0 \\ 0 & 1 & 0 & 1 \\ 1 & 0 & 1 & 0} \in \Mat_4\prs{\mbb{C}}\]
מיצאו מטריצה
$P \in \Mat_4\prs{\mbb{C}}$
עבורה
\[\text{.} P^t A P = \pmat{1 & 0 & 0 & 0 \\ 0 & 1 & 0 & 0 \\ 0 & 0 & 0 & 0 \\ 0 & 0 & 0 & 0}\]
\end{exercise}

\begin{exercise}
תהי
\[\text{.} A = \pmat{1 & 1 & 1 \\ 1 & 1 & -1 \\ -2 & 1 & 0} \in \Mat_3\prs{\mbb{R}}\]

\begin{enumerate}
\item הראו כי
\[\text{.} \forall v \in \mbb{R}^3 \colon \sqrt{2} \norm{v} \leq \norm{Av} \leq \sqrt{6} \norm{v}\]

\item עבור
$\alpha \in \brs{0,1}$
חשבו את
$\norm{A \pmat{\alpha \\ \sqrt{1 - \alpha^2} \\ 0}}$.

\item מצאו וקטור
$v \in \mbb{R}^3$
עבורו
$\norm{v} = 1$
וגם
$\norm{Av} = 2$.
\end{enumerate}
\end{exercise}

\begin{exercise}
יהי
$V$
מרחב מכפלה פנימית סוף־מימדי מעל
$\mbb{C}$
ותהי
$T \colon V \to V$.

\begin{enumerate}
\item הראו כי
$\ker\prs{T^*} = \im\prs{T}^\perp$.
\item הראו כי
$\im\prs{T^*} = \ker\prs{T}^\perp$.
\item
נניח כי
$T^2 = T$.
הראו כי
$T$
הטלה אורתוגונלית אם ורק אם
$T$
הרמיטית.
\end{enumerate}
\end{exercise}

\end{document}