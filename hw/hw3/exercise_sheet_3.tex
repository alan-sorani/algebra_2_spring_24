\documentclass[a4paper,10pt,twoside,openany]{article}

\usepackage[lang=hebrew]{maths}
\usepackage{hebrewdoc}
\usepackage{stylish}
\usepackage{lipsum}
\let\bs\blacksquare

\setlength{\parindent}{0pt}

%%%%%%%%%%%%
% Styling %
%%%%%%%%%%%%

\usepackage{enumitem}

%%%%%%%%%%%%%
% Counters  %
%%%%%%%%%%%%%

\setcounter{section}{1}     
            
%BIBLIOGRAPHY
\usepackage[
backend=biber,
style=alphabetic,
]{biblatex}
\addbibresource{bibliography.bib} %Imports bibliography file

%%%%%%%%%%
% Title  %
%%%%%%%%%%
\title{
אלגברה ב' - גיליון תרגילי בית 3 \\
משפט ז'ורדן, ומרחבי מכפלה פנימית
\\
\vspace{1cm}
\large{תאריך הגשה: 1.8.2024}
}
\date{}

\begin{document}
\maketitle

\begin{exercise}%1
מיצאו בסיס מז'רדן לכל אחד מהאופרטורים הבאים.

\begin{enumerate}
\item
$T_A$
כאשר
\[\text{.} A = \pmat{7&1&-8&-1\\0&3&0&0\\4&2&-5&1\\0&-4&0&-1} \in \Mat_4\prs{\mbb{C}}\]

\item
$T_A$
כאשר
\[A=\pmat{1 & 2 & 2 & 1 \\ 2 & - 1 & -3 & -2 \\ -2 & 3 & 5 & 2 \\ -1 & 2 & 2 & 3} \in \Mat_4\prs{\mbb{C}}\]
ונתון כי
$2$
ערך עצמי יחיד.

\item
\begin{align*}
T \colon \Mat_3\prs{\mbb{C}} &\to \Mat_3\prs{\mbb{C}} \\
A &\mapsto J_3\prs{0} A
\end{align*}

\item
\begin{align*}
T \colon \mbb{C}_3\brs{x} &\to \mbb{C}_3\brs{x} \\
T\prs{p}\prs{x} &= p\prs{x+1}
\end{align*}
\end{enumerate}
\end{exercise}

\begin{exercise}%2
\begin{enumerate}
\item מצאו את צורת ז'ורדן של
$J_n\prs{\lambda}^{-1}$
עבור
$\lambda \in \mbb{C}\setminus\set{0}$
ועבור
$n \in \mbb{N}_+$.

\item תהי
$A \in M_n\prs{\mbb{C}}$
הפיכה.
מצאו את צורת ז'ורדן של
$A^{-1}$.

\item מצאו תנאי הכרחי ומספיק על
$A \in M_n\prs{\mbb{C}}$
כך שיתקיים
$A \sim A^{-1}$.
\end{enumerate}
\end{exercise}

\begin{exercise}
ראינו בתרגול שאם
$A = P J P^{-1}$
אז
$A^k = P J^k P^{-1}$
לכל
$k \in \mbb{N}$.
לכן, כדי לחשב חזקות של מטריצה די למצוא בסיס ז'ורדן של ההעתקה המתאימה ולדעת לחשב חזקות של צורת ז'ורדן.

\begin{enumerate}
\item יהיו
$n,k \in \mbb{N}_+$.
חשבו את
$J_n\prs{0}^k$.

\item
מתקיים
$J_n\prs{\lambda} = J_n\prs{0} + \lambda I_n$.
היעזרו בכך ובסעיף הקודם כדי למצוא נוסחא כללית עבור
$J_n\prs{\lambda}^n$
לכל
$\lambda \in \mbb{F}$
ולכל
$n,j \in \mbb{N}_+$.
\end{enumerate}
\end{exercise}

\begin{exercise}
בתרגיל זה נראה כיצד צורת ז'ורדן עוזרת בחישוב בעיות המצריכות חזקות של מטריצות.

\begin{enumerate}
\item
תהיינה
$A,B,P \in \Mat_n\prs{\mbb{C}}$
כאשר
$P$
הפיכה וגם
$A = P^{-1} B P$.
הראו כי
$A^r = P^{-1} B^r P$
לכל
$r \in \mbb{N}$.

\item
בשמורת הטבע ליד הטכניון סין יש היום 2 דרקונים, 600 פנדות ו־20000 במבוקים.

כל שנה הדרקונים, הפנדות והבמבוקים מתרבים ומספרם גדל פי 2.

לאחר מכן, כל פנדה אוכלת במבוק אחד וכל דרקון אוכל שתי פנדות.

אז, רשות הטבע והגנים הסינית משחררת לטבע 4 דרקונים ו־10 פנדות, אם עדיין יש פנדות בשמורה.

לבסוף, אם לא נשאר במבוק בסוף השנה, כל הפנדות מתות.

\begin{enumerate}
\item
מיצאו מטריצה
$A \in \Mat_4\prs{\mbb{C}}$
וערכים
$d,p,b$
עבורם מספרי הדרקונים, הפנדות והבמבוקים בסוף השנה ה־%
$t$
הם מקדמים בוקטור
$A^t \pmat{1 \\ d \\ p \\ b}$
לכל
$t \in \mbb{N} \cup \set{0}$.

\item נשיא הטכניון מתכנן לבקר בסין עוד 30 שנה. האם יהיו פנדות בשמורה בזמן הביקור שלו?

\item הטכניון החליט להעביר את הלימודים מסין למאדים עוד 230 שנה. האם ישארו עד אז פנדות בשמורת הטבע?
\end{enumerate}
\end{enumerate}
\end{exercise}

\begin{exercise}
עבור ההעתקות הבאות $f_i$, קיבעו האם $f_i$ מכפלה פנימית.

\begin{enumerate}
\item
\begin{align*}
f_1 \colon \mbb{R}^3 \times \mbb{R}^3 &\to \mbb{R} \\
\prs{\pmat{a\\b\\c}, \pmat{x\\y\\z}} &\mapsto ax + by + az
\end{align*}

\item
\begin{align*}
f_2 \colon \mbb{R}^3 \times \mbb{R}^3 &\to \mbb{R} \\
\prs{\pmat{a\\b\\c}, \pmat{x\\y\\z}} &\mapsto ax + by + cz + xz
\end{align*}

\item
\begin{align*}
f_3 \colon \Mat_n\prs{\mbb{C}} \times \Mat_n\prs{\mbb{C}} &\to \mbb{C} \\
\prs{A,B} &\mapsto \tr\prs{B^t A}
\end{align*}

\item
\begin{align*}
f_4 \colon \mbb{R}_n\brs{x} \times \mbb{R}_n\brs{x} &\to \mbb{R} \\
\prs{f,g} &\mapsto f\prs{0} g\prs{0} + \ldots + f\prs{n} g\prs{n}
\end{align*}

\item
\begin{align*}
f_5 \colon \mbb{C}_n\brs{x} \times \mbb{C}_n\brs{x} &\to \mbb{C} \\
\prs{f,g} &\mapsto f\prs{0} g\prs{0} + \ldots + f\prs{n} g\prs{n}
\end{align*}
\end{enumerate}
\end{exercise}

\begin{exercise}
היעזרו באי־שוויון קושי־שוורץ כדי להראות שמתקיים
\[\text{.} \forall x,y,z \in \mbb{R}_+ \colon x + y + z \leq 2 \prs{\frac{x^2}{y+z} + \frac{y^2}{x+z} + \frac{z^2}{x+y}}\]
\textbf{רמז:}
חישבו כיצד לפרש את אגף ימין בעזרת נורמה.
\end{exercise}

\end{document}